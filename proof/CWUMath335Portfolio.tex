\documentclass[12pt]{extarticle}
%Some packages I commonly use.
\usepackage[english]{babel}
\usepackage{graphicx}
\usepackage{framed}
\usepackage[normalem]{ulem}
\usepackage{amsmath}
\usepackage{amsthm}
\usepackage{amssymb}
\usepackage{amsfonts}
\usepackage{enumerate}
\usepackage[utf8]{inputenc}
\usepackage[top=1 in,bottom=1in, left=1 in, right=1 in]{geometry}

%A bunch of definitions that make my life easier
\newcommand{\matlab}{{\sc Matlab} }
\newcommand{\cvec}[1]{{\mathbf #1}}
\newcommand{\rvec}[1]{\vec{\mathbf #1}}
\newcommand{\ihat}{\hat{\textbf{\i}}}
\newcommand{\jhat}{\hat{\textbf{\j}}}
\newcommand{\khat}{\hat{\textbf{k}}}
\newcommand{\minor}{{\rm minor}}
\newcommand{\trace}{{\rm trace}}
\newcommand{\spn}{{\rm Span}}
\newcommand{\rem}{{\rm rem}}
\newcommand{\ran}{{\rm range}}
\newcommand{\range}{{\rm range}}
\newcommand{\mdiv}{{\rm div}}
\newcommand{\proj}{{\rm proj}}
\newcommand{\R}{\mathbb{R}}
\newcommand{\N}{\mathbb{N}}
\newcommand{\Q}{\mathbb{Q}}
\newcommand{\Z}{\mathbb{Z}}
\newcommand{\<}{\langle}
\renewcommand{\>}{\rangle}
\renewcommand{\emptyset}{\varnothing}
\newcommand{\attn}[1]{\textbf{#1}}
\theoremstyle{definition}
\newtheorem{theorem}{Theorem}
\newtheorem{corollary}{Corollary}
\newtheorem*{definition}{Definition}
\newtheorem*{example}{Example}
\newtheorem*{note}{Note}
\newtheorem{exercise}{Exercise}
\newcommand{\bproof}{\bigskip {\bf Proof. }}
\newcommand{\eproof}{\hfill\qedsymbol}
\newcommand{\Disp}{\displaystyle}
\newcommand{\qe}{\hfill\(\bigtriangledown\)}
\setlength{\columnseprule}{1 pt}


\title{Math 335 Portfolio}
\author{Jean Marie Linhart}
\date{January 2019}

\begin{document}

\maketitle

\section{Induction Proofs}
\subsection{Ordinary Induction}
\begin{exercise} Prove, for all natural numbers $n$, that 
\begin{equation} \sum_{k=0}^n k = 1 + 2 + 3 + \cdots + n = \dfrac{n(n+1)}{2}
\label{eq:Pn}\end{equation}
\end{exercise}

\begin{proof}
We prove this by induction on $n\in\N$.  In the base case, $n=0$, and \eqref{eq:Pn} becomes
$$\sum_{k=0}^n k = \sum_{k=0}^0 k = 0 = \dfrac{0(1)}{2} = \dfrac{n(n+1)}{2}$$

Now, let $n>0$ be arbitrary, and assume \eqref{eq:Pn}.
We show $\displaystyle \sum_{k=0}^{n+1} k = \dfrac{(n+1)(n+2)}{2}$.
To that end note 
\begin{align*}
	\sum_{k=0}^{n+1} k &= \left(\sum_{k=0}^{n} k\right) + (n+1) &\mbox{(sum definition)}\\
	&= \frac{n(n+1)}{2} + (n+1) &\mbox{(induction hypothesis)}
	\\
	&= \frac{n(n+1)}{2} + \frac{2n+2}{2} &\mbox{(common denominator)}
	\\
	&= \frac{n^2 +n}{2} + \frac{2n+2}{2} &\mbox{(distribute)}
	\\
	&= \frac{n^2 +3n + 2}{2} &\mbox{(combine like terms)}
	\\
	&= \frac{(n+1)(n+2)}{2} & \mbox{(factor the numerator)}\\
\end{align*}

In all cases, \eqref{eq:Pn} is true, so $\forall n\in \N$, 
$\Disp \sum_{k=0}^n k = \dfrac{n(n+1)}{2}$
\end{proof}



\end{document}
